In~\cite{2013weka}, the authors use WEKA data mining tool to determine if a movie will be a success, neutral or flop in the box office. They combine data gathered from Social Media with factors like the actors and director popularity. This paper only presents a final classification (flop, success and neutral) and does not provide visualization tools for detailed results. They also provide some conclusions about which factors they found were the most relevant ones.

Paper~\cite{VisualizationMovie} uses visualization techniques to find hidden relationships between movies and their reviews. The authors, also, analyze the word of mouth effects on movies profits. This paper does not provide sentiment analysis it only analyzes the ratio of reviews over time. In this paper, the authors could also find some attempts to manipulate reviews rates to manipulate people into watching a particular movie.

In paper~\cite{2014Influencesocialmedia}, the authors investigate if Social Media does significant influences a movie performance. They collect data from Youtube, Twitter, and Facebook to verify if this data can impact the movie's profits. On their findings, they stated that some features are very useful to improve the prediction, but others such as Facebook likes are not reliable. They also conclude that the popularity of actors involved in the movie is one key factor for revenue.

Papers~\cite{2015Crowd-Source} and~\cite{2015DifferentFactors} combine the classical features (actors, directors, script..) used to calculate movies revenue, with features extracted from social media (Youtube comments and tweets) to predict the movie gross. Both papers stated that using classical and social media features combined improve the prediction accuracy precision.

Paper~\cite{2016EmotionsIMDB} is the first one we found, that focus on the movie's viewers and not on the movie's makers. They collected movie reviews from IMDB movie site, and evaluate them according to 4 dimensions: pleasantness, attention, sensitivity, and aptitude. They present their findings in an Emotion Map, which is a heatmap, that shows the number of reviews in each dimension.

In~\cite{2016LSISVD}, the authors predict the users' preferred movies genre and recommending movies according to the predict genre. They collected data from Twitter and used LSI and LDA techniques for the prediction. This paper does not present any visualization implementation their model only presents the movies genres and movie recommendations. 
 
In \cite{2017MoviePredictionData}, the authors used a data mining model to find out the success rating of a movie. The model proposed by them consider factors like Movie Name, Year of release, Genres (Drama, Action, Romance, Comedy, Other), Directors, Music directors, Producers and Languages and uses historical data for the prediction.

The authors of paper \cite{2017MiningYouTubeTrailers} extract features like number of views, number of likes and dislikes from Youtube movies trailers. Then they use these features to predict the gross incoming of the movies through data mining algorithms. They tested several data mining algorithms to find one with better accuracy. The sentiment analysis of the comments is mentioned only as future work for this paper.

Paper \cite{2017PredictingMarketRevenue} presents a prediction model for movie revenue on its first weekend. This paper uses tweets collected on the opening week of a movie first week and, through machine learning algorithms, predict the gross amount for the movie during the first weekend. Several techniques were combined to find out the most accurate results.