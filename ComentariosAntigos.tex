%Comentarios Introdução

%Every year the movie industry releases thousands of movies, and along with them, massive marketing campaigns. These marketing campaigns are used to attract viewers to the theaters and increase the movie revenue. However, them alone are not enough to grant the success of a movie. Besides that, the viewers need to enjoy the film for it to be successful.  

%Several studies have been written regarding movie profit, trying to predict how well a film will perform at the box office and how much revenue it will generate. But on these type of analysis, one major person is often put aside, the viewer.  

%JS ajuste de inglês
%The viewer usually goes to the movie theater with high expectations about a film, but these expectations %quite 
%The viewer uses to go to the movie theater with high expectations about a film, but these expectations %quite 
%sometimes are not matched. In these cases, the viewer spends time and money to watch a film, and, ends up disappointed.  
%SO: complementando aqui
%These disappointments can generate doubts when deciding whether going out to watch new films. 
%The viewer goes to the theater with high expectations for a film, but these expectations quite sometimes are not matched. On these cases, the viewer spends time and money on a film, and, ends up disappointed. 

%Internet provides several sites with movies reviews that can be checked before going to the theater. But these sites contain lots of opinions, and to read them all takes a long time. Also, the critic review is often not reliable, since several movies get bad reviews by the critic, but are loved by the regular viewer.  

%JS Ajustes
%In order to help to identify the public opinion about a movie, we develop a new approach for the visual analysis of viewers' comments on social networks about a movie. It synthesizes the opinions gathered from Twitter and YouTube, showing the most relevant aspects (good and bad) that are being discussed in the social networks. In our case studies, we provide charts showing the opinions before and after three blockbuster films being released.

%For the purpose of identifying the public opinion concerning a movie, we present a new approach for visual analysis of viewer's comments on social networks. It synthesizes the opinions gathered from Twitter and YouTube, highlighting the most relevant aspects (good and bad) that are being discussed in those platforms. In our case-studies, we provide charts displaying the before and after opinions related to the release of three blockbuster films.
% I: sugestão de reescrever o parágrafo acima, já incluindo "a ideia" do primeiro parágrafo da atual seção II. Research Questions:
%  For the purpose of identifying the public opinion concerning a movie, we present a new approach for the visual analysis of viewers comments on social networks. It synthesizes the opinions gathered from Twitter and YouTube comments about movies, including Sentiment Analysis results, and highlights the most relevant aspects being discussed in those platforms to identify the degree of public satisfaction. We present three case studies with the analysis of three blockbuster movies released in 2018 and 2019. For each one, we provide charts displaying public opinion before and after its release.
% JS fix um mix das duas versões

%In order to help the identification of public opinion about a movie, we present a new approach for the visual analysis of the viewers' comments on social networks about a movie. It synthesizes the opinions gathered from Twitter and YouTube, showing the most relevant aspects (good and bad) that are being discussed in the social networks. In our case studies, we provide charts showing the opinions before and after three blockbuster films being released.  
%In this work, we proposed an internet site where the user can quickly check what is being said about the movie he wants to watch. The site will synthesize the opinions, gathered from Twitter and YouTube, and show the user the most relevant aspects (good and bad) that are being said about a movie. The system will, also, provide graphics showing the opinions before and after the movie released.  



%The public is one of the key factors for the success of a movie. To deal with public opinion, we developed some sentiment analysis tools that provided features like:  The general opinion from the public about the three selected movies (Avengers Infinite War, Aquaman, and Captain Marvel); Identification of main used words on the evaluated texts through the use of TF-IDF statistics; A "recommendation text" created using the most frequent adjectives obtained through TF-IDF use.
%So, our work focus in mining public opinion to verify the public reception of a movie. 
%Our goal is to provide a general opinion from the public, about 

%\begin{itemize}
%    \item General opinion from public (tweets), about movies: Avengers Infinite War, Aquaman and Captain Marvel
%    \item Term Frequency: Use TF-IDF to identify the main terms in tweets and comments
%    \item Part of Speeach (POS): Derive movie recommendations based on most used adjectives obtained by TF-IDF features.
%\end{itemize}
%We wanted to create a tool that would provide

%Inthis approach, the following contributions will be presented: 
%This analysis utilizes public opinion to be aware the satisfaction degree. In this research, we study the following research questions:
%Our study, we revolves arround 2 main research que

%In this study, we will present the following contributions.

%\begin{itemize}
%    \item RQ1: What is the possibility of creating a critical opinion from tweets?
%    \item RQ2: Can the opinion impact the profits of the movie?
%\end{itemize}

%We chose these questions because, we believe they represents a main factor, for the future blockbusters productions studios. In other wordsm, the public are the key role in the movie success. 

%\begin{itemize}
%    \item General opinion from public (tweets), about movies: Avengers Infinite War, Aquaman and Captain Marvel
%    \item Term Frequency: Use TF-IDF to identify the main terms in tweets and comments
%    \item Part of Speeach (POS): Derive movie recommendations based on most used adjectives obtained by TF-IDF features.
%\end{itemize}