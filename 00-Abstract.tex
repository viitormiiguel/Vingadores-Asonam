%SO: comecei aqui...
%{\color{red}TODO}
Nowadays, massive marketing campaigns are done before and after the release of blockbuster movies. However, just a marketing campaign is not enough to grant the success of a movie.  Several research papers have been done regarding movie profit, trying to predict how well it will perform at the box office. However, usually, the viewer ability to propagate good or bad comments about a movie through social media is put aside. This paper presents a new approach for visual analysis of viewers comments on social media with the main goal to find out a way to compute the collective mind about a certain movie. Through the application of an unsupervised lexicon-based approach, we generate a ``recommendation text'' based on the opinion expressed on Twitter and YouTube, and several visualizations that allow comparing the total number of interaction, the most frequent terms, and the adjective used for the main characters. In order to validate our model, we selected three blockbuster movies released in 2018 and 2019 as a case study.
% This paper presents a model to process, classify and organize public opinions regarding specifically recent blockbusters movies. The area we want to contribute is in the film recommendation, but our method can be applied in other areas as well. Our hypothesis is that people can be informed about film critics and generic people posts but it is hard to know the collective idea about a certain movie production. It is the main goal of this paper, i.e. to find out a way to compute the collective mind about a certain movie. results indicate that...
%SO: pensando... seria legal avaliarmos com um grupo focal (para ser possivel de realizar) se o nosso "colective mind" sobre um determinado filme melhor representa o filme do que algumas criticas de experts e posts individuais. Por isso tinha que ser testado com pessoas que viram o filme...