
We present a new approach for the visual analysis of viewers comments on two social media: Twitter and YouTube. It analyzes and synthesizes the public opinion concerning a movie, highlighting what are being discussed and the public satisfaction in those platforms. We showed the obtained results through three case studies where we analyze three blockbuster movies, displaying public opinion before and after its release.

Our contribution is mainly the generation of a ``recommendation text'' through the application of an unsupervised lexicon-based approach, which represents the public opinion about a certain movie.  Another contribution is that our approach allows a visual analysis of the collected data showing the total number of interaction, the most frequent terms, and the adjective used for the main characters. 

% I: "inventei" uma atividade para dizer que estamos fazendo "agora"... 
We are now working in the implementation of an interactive interface to facilitate the visual analysis. For our future works, we intend to analyze the use of \textit{Dirty Words}, to identify if there is an intensification of the sentiment expressed in previous classification. 