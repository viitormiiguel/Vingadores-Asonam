%JS
%Limpando comentarios antigos para facilitar a leitura do texto.
%Os comentarios poderao ser encontrados no arquivo ComentariosAntigos.tex
Every year the movie industry releases thousands of movies, and along with them, massive marketing campaigns. Those marketing campaigns are used to attract viewers to the movie theaters and increase movie revenue. However, just a marketing campaign is not enough to grant the success of a movie. Besides that, the viewers need to enjoy, comment and recommend movies to ensure its success and a big box office. 

Several studies have been written regarding movie profit, trying to predict how well it will perform at the box office and how much revenue it will generate~\cite{2015DifferentFactors}~\cite{2014Influencesocialmedia}. However, usually in this type of analysis, the viewer ability to propagate good or bad comments about a movie, through social media, is often put aside.

%JS: Revisão by Pedro
Although the regular viewer tends to go to the movie theater with high expectations towards a movie, those expectations are occasionally not matched~\cite{2016EmotionsIMDB}. In such cases, the viewer spends time and money to watch a film and ends up disappointed. Experiencing those disappointments may generate apprehension for the viewer when deciding whether going out to watch new movies or not.

%JS: Revisão by Pedro
Checking for movie review websites before going to a movie theater is a fairly common practice among viewers~\cite{2016EmotionsIMDB}. Given the sheer amount of users' and critics' reviews those websites usually present, reading through all that information can be a very time-consuming task. Moreover, reading only the critics' review may not be a reliable method, considering the divergences that often arise between critic and general public reviews.
%SO: Aqui seria bom ter uma ref
% I: Também acho que precisa uma referência no final do parágrafo acima.

% I: Importante:
% 1) Verificar se o formato está correto para a conferência, já que tem um "recado no título".
%JS O Template está correto vou reitirar está parte do comentario

% 2) Confirmar se vai ficar o título que eu sugeri ou se será alterado.
% 3) Complementar nossos dados colocando "School of Technology, PUCRS - Pontifícia Universidade Católica do Rio Grande do Sul, Porto Alegre, Brazil".
% 4) Colocar agradecimentos. No caso da Joana precisa o seguinte:
% This work was achieved in cooperation with HP Brasil Indústria e Comércio de Equipamentos Eletrônicos LTDA. using incentives of Brazilian Informatics Law (Law nº 8.2.48 of 1991). The authors would like to thank the financial support of the PDTI Program, financed by Dell Computers of Brazil Ltd. (Law 8.248/91). The authors also would like to thank the financial support of Progress Informática Ltda.

For the purpose of identifying the public opinion concerning a movie, we present a new approach for visual analysis of viewer's comments on social media. It synthesizes the opinions gathered from Twitter and YouTube, highlighting the most relevant aspects (good and bad) that are being discussed in those platforms. Our approach allowed us to identify the degree of public satisfaction about a movie. We present three case studies where we analyze three blockbuster movies released in 2018 and 2019. For each one, we provide charts displaying public opinion before and after its release.

%Versão1
%The main contributions of this paper are: 
%\begin{itemize}
%    \item Development of an unsupervised lexicon-based approach to compare opinions from Twitter and YouTube;
%    \item Extraction of most frequent terms from YouTube comments and tweets; and
%    \item Generation of a "recommendation text", generated by opinions collected on Twitter and YouTube.
    %\item Generate textual recommendation based on the opinion of twitter and youtube users
%\end{itemize}

%revisar ultima frase
% I: revisei e fiz pequenas alterações de inglês. A original está em comentário.
The main contribution of this paper is the application of an unsupervised lexicon-based approach to generate a  ``recommendation text'', based on the opinion expressed on social media (Twitter and YouTube). We also provide visualization techniques that allow comparing the total number of interaction, the most frequent terms, and the adjectives used for the main characters.
%The main contribution of this paper is the application of an unsupervised lexicon-based approach to generate a "recommendation text", based on  opinion expressed on social media (Twitter and YouTube). We also provide visualization techniques that allows to compare total number of interaction, the most frequent terms, and the adjective used for the main characters.

% REVISAR PARÁGRAFO ABAIXO NO FINAL!!!!
The remainder of this paper is organized as follows. Related work is presented in Section~\ref{sec:related}. Section~\ref{sec:proposed} describes the proposed approach along with the used methodology.
The achieved results are presented in Section~\ref{sec:results} through the analysis of three case studies. Finally, conclusions and goals for future research come in Section~\ref{sec:conclusions}. 
